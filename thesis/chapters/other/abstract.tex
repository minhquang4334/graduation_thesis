Con người hiếm khi giải quyết vấn đề mà không có chút hiểu biết, kiến thức gì về vấn đề đó. Quan sát này chính là động lực trong việc xây dựng thuật toán tối ưu đa nhiệm thông qua việc trao đổi tri thức giữa các bài toán có liên quan đến nhau. Trong đó các bài toán sẽ được coi như là các tác vụ có thể giải quyết đồng thời bằng thuật toán tối ưu hóa đa nhiệm. Các nghiệm tốt giữa các tác vụ có thể được trao đổi lẫn nhau để cải thiện hiệu suất trên từng tác vụ. Tuy nhiên khi chưa có bất kỳ một kiến thức nào biết trước về mối quan hệ giữa các task, thì việc trao đổi nghiệm rất có thể sẽ dẫn đến "trao đổi âm" (negative transfer). Việc xảy ra "trao đổi âm" sẽ dẫn đến giảm tốc độ hội tụ trên tất cả các tác vụ. Đây cũng là một vấn đề mà thuật toán tối ưu hóa đa nhiệm phổ biến là Multifactorial Evolutionary (MFEA-I) gặp phải. 

Vậy nên thuật toán Multifactorial Evolutionary II (MFEA-II - phiên bản tiếp theo thuật toán MFEA-I) đã ra đời để giải quyết được vấn đề của phiên bản trước. Đây là thuật toán tính toán tiến hóa mới nhất cho phép học được mối quan hệ giữa các tác vụ dựa trực tiếp vào dữ liệu sinh ra trong quá trình tối ưu, từ đó khai thác sự bổ trợ giữa các tác vụ một cách hiệu quả hơn. Giống như giải thuật tiền nhiệm là MFEA-I, MFEA-II cũng có khả năng huấn luyện các mạng Nơ-ron có giám sát tốt hơn các giải thuật tiến hóa đơn nhiệm thông thường. Thêm nữa thuật toán tiến hóa đa nhiệm cũng có nhiều lợi thế trong các bài toán học tăng cường, bởi khi gradient của bộ tham số tương đối nhiễu, việc xác định chính xác giá trị gradient gặp nhiều khó khăn.
Tuy vậy, trong tầm hiểu biết của tôi, chưa có nghiên cứu nào trong việc áp dụng thuật toán MFEA-II để huấn luyện nhiều mạng Nơ-ron có cấu trúc (modular neural network) và huấn luyện nhiều mạng nơ-ron trong học tăng cường.  
Vậy nên sau quá trình đào sâu nghiên cứu thuật toán tối ưu hóa đa nhiệm tôi đề xuất xây dựng một bộ thư viện để áp dụng thuật toán MFEA-II giải quyết bài toán nhiều mạng Nơ-ron có cấu trúc và huấn luyến nhiều mạng nơ-ron cho các môi trường học tăng cường có liên quan đến nhau.

Đồ án của tôi được xây dựng như sau:
\begin{itemize}
  \item \textbf{Chapter 1} Cung cấp thông tin, khái niệm tổng quan về thuật toán tiến hóa, cơ bản về thuật toán MFEA-II
  \item \textbf{Chapter 2} Giới thiệu về bài toán huấn luyện mạng Neural có cấu trúc Mô-đun (modular neural network) và huấn luyện mạng nơ ron trong học tăng cường
  \item \textbf{Chapter 3} Đề xuất hướng áp dụng thuật toán MFEA-II cho việc huấn luyện nhiều mạng Nơ-Ron có cấu trúc mô-đun với các lớp ẩn khác nhau và huấn luyện nhiều mạng nơ-ron cho các môi trường học tăng cường có liên quan đến nhau.
  \item \textbf{Chapter 4} Trình bày kết quả thực nghiệm, phân tích và đánh giá hiệu quả của giải thuật đã đề xuất
\end{itemize}

\pagebreak